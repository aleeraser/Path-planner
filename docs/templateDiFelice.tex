\documentclass[conference]{IEEEtran}
\makeatletter
\def\ps@headings{%
\def\@oddhead{\mbox{}\scriptsize\rightmark \hfil \thepage}%
\def\@evenhead{\scriptsize\thepage \hfil \leftmark\mbox{}}%
\def\@oddfoot{}%
\def\@evenfoot{}}
\makeatother
\pagestyle{headings}

\hyphenation{op-tical net-works semi-conduc-tor}


\usepackage{subfigure}
\usepackage{soul}

\usepackage{algorithmic}
\usepackage[ruled,vlined]{algorithm2e}

\ifCLASSINFOpdf
  \usepackage[pdftex]{graphicx}
  % declare the path(s) where your graphic files are
  % \graphicspath{{../pdf/}{../jpeg/}}
  % and their extensions so you won't have to specify these with
  % every instance of \includegraphics
  % \DeclareGraphicsExtensions{.pdf,.jpeg,.png}
\else
  % or other class option (dvipsone, dvipdf, if not using dvips). graphicx
  % will default to the driver specified in the system graphics.cfg if no
  % driver is specified.
  % \usepackage[dvips]{graphicx}
  % declare the path(s) where your graphic files are
  % \graphicspath{{../eps/}}
  % and their extensions so you won't have to specify these with
  % every instance of \includegraphics
  % \DeclareGraphicsExtensions{.eps}
\fi
% graphicx was written by David Carlisle and Sebastian Rahtz. It is
% required if you want graphics, photos, etc. graphicx.sty is already
% installed on most LaTeX systems. The latest version and documentation can
% be obtained at:
% http://www.ctan.org/tex-archive/macros/latex/required/graphics/
% Another good source of documentation is "Using Imported Graphics in
% LaTeX2e" by Keith Reckdahl which can be found as epslatex.ps or
% epslatex.pdf at: http://www.ctan.org/tex-archive/info/
%
% latex, and pdflatex in dvi mode, support graphics in encapsulated
% postscript (.eps) format. pdflatex in pdf mode supports graphics
% in .pdf, .jpeg, .png and .mps (metapost) formats. Users should ensure
% that all non-photo figures use a vector format (.eps, .pdf, .mps) and
% not a bitmapped formats (.jpeg, .png). IEEE frowns on bitmapped formats
% which can result in "jaggedy"/blurry rendering of lines and letters as
% well as large increases in file sizes.
%
% You can find documentation about the pdfTeX application at:
% http://www.tug.org/applications/pdftex

\begin{document}
\title{Implementazione ed analisi \\ delle principali tecniche di path planning}

% Author names 
% note positions of commas and nonbreaking spaces ( ~ ) LaTeX will not break
% a structure at a ~ so this keeps an author's name from being broken across
% two lines.
% use \thanks{} to gain access to the first footnote area
% a separate \thanks must be used for each paragraph as LaTeX2e's \thanks
% was not built to handle multiple paragraphs
\author{
Filippo Morselli\IEEEauthorrefmark{1},
Carlo Stomeo\IEEEauthorrefmark{1},
Alessandro Zini\IEEEauthorrefmark{1}
\\
\IEEEauthorblockA{\IEEEauthorrefmark{1} DISI, University of Bologna, Italy \\
 \\
Emails: filippo.morselli@studio.unibo.it, carlo.stomeo@studio.unibo.it, alessandro.zini@studio.unibo.it}}




% make the title area
\maketitle

% The Abstract
\begin{abstract}
% Riassumi qui il succo del progetto, in 20-30 righe.
Il presente documento descrive il lavoro realizzato per il progetto del corso di Sistemi Mobili.
Le attivit\'a svolte rappresentano la prima met\'a di un progetto pi\'u ampio, che ha come scopo
l'implementazione di algoritmi di mobilit\'a in un mondo fisico con ostacoli per un rover dotato
di sensori.
\\
Per questa fase \'e stata realizzata una libreria Javascript che mette a disposizione una
piattaforma generica per il posizionamento ed il movimento di oggetti in un ambiente bidimensionale
suddiviso in celle, e a partire da essa sono state implementate e testate le principali metodologie
di path planning trattate durante le lezioni del corso di Sistemi Mobili. 
\\
Il prodotto finale della cosegna consiste dunque in una applicazione web che realizza un ambiente
personalizzabile dall'utente in dimensioni, ostacoli e punti di partenza e destinazione.
Nell'ambiente creato si pu\'o poi ottenere il percorso per raggiungere la destinazione, 
selezionando uno fra i diversi algoritmi di implementati.
\\
L'obiettivo \'e quello di mettere in pratica ed approfondire fino al dettaglio implementativo le
metodologie di path planning studiate a lezione, e di avere un software personalizzabile e 
facilmente estendibile che funga da solido punto di partenza per la seconda met\'a del progetto,
ossia l'effettiva mobilit\'a nel mondo fisico.
\end{abstract}



\section{Introduction}
% Descrivi qui il contesto generale, i contributi del progetto, e la struttura del documento.
Il \textit{path planning} \'e il processo di determinare il percorso che una 
entit\'a mobile deve seguire per raggiungere una certa destinazione. Tuttavia \'e una tipologia 
di problema che presenta numerose varianti, deteriminate dalle caratteristiche dell'ambiente
in cui avviene la mobilit\'a e dalle capacit\'a e conoscenze dell'oggetto mobile. 
\\
Il contesto preso in considerazione durante la realizzazione del progetto assume un ambiente
bidimensionale con una suddivisione a griglia del piano. Ciascuna cella del piano può essere libera
o occupata da un ostacolo. Due celle libere sono contrassegnate come punto di partenza e punto di
destinazione e il problema consiste nel trovare il miglior percorso che li colleghi.
\\
Si assume che le dimensioni dell'oggetto mobile siano sempre contenute all'interno della cella
unitaria dell'ambiente, quindi il percorso \'e definito come una sequenza di celle libere 
adiacenti (anche in diagonale, purch\'e non si passi fra due ostacoli) che porti dalla partenza 
alla destinazione. 
L'ambiente \'e di tipo statico, ossia non si hanno modifiche alla configurazione degli ostacoli
una volta inizate le computazioni per la creazione del percorso.
\\
In un ambiente definito in questo modo, si sono studiate due versioni del problema: il 
\textit{global} o \textit{offline path planning} in cui si ha una completa conoscenza a priori
della configurazione dell'ambiente, e il \textit{local} o \textit{online path planning} in cui gli
unici input sono la partenza e la destinazione e l'entit\'a mobile scopre la presenza di ostacoli
solamente quando \'e in prossimit\'a di essi, simulando la presenza di sensori.
\\
\\ 
Per entrambi i problemi si \'e voluto implementare le tecniche maggiormente utilizzate, studiarne
il comportamento per diverse configurazioni dell'ambiente e confrontarne le prestazioni.
Per raggiungere questo obiettivo \'e stato prima necessario sviluppare un modulo software che 
mette a disposizione l'effettivo ambiente e che fornisce
l'interfaccia per interagire con esso. Il modulo in questione \'e una libreria Javascript che
permette la creazione della griglia, il posizionamento e il movimento di oggetti su di essa.
L'interazione con l'ambiente cos\'i creato \'e possibile anche tramite pagina web, grazie a una
interfaccia grafica realizzata usando canvas HTML, che consente il posizionamento degli ostacoli
e la visualizzazione del percorso generato dagli algoritmi.
\\
Le tecniche implementate sono le seguenti:
\begin{itemize}
\item Cellular decomposition
\item Visibility graph
\item Probabilistic Roadmap
\item Potential fields
\item Bug (versione 1 e 2)
\item Tangent bug \\~\
\end{itemize}

Il resto del documento \'e strutturato nel seguente modo. 


\section{Related works}
Fornisci una breve rassegna di articoli di ricerca, software, prototipi o tecnologie che sono collegate in qualche modo al problema affrontato nel progetto. Tutti i lavori devono essere referenziati ed inseriti nella Bibliografia.

\section{Architecture}
Struttura concettuale del software che e\' stato sviluppato; può essere utile inserire almeno un'architettura delle componenti del sistema.

\section{Implementation}
Descrivi come e\' stato implementato il sistema, ossia tecnologie utilizzate, linguaggi, APIs, etc. Nel caso, fornisci pseudo-codice degli algoritmi
piu\' interessanti sviluppati nel progetto.

\section{Performance evaluation}
Illustra qui i risultati sperimentali (simulazioni o esperimenti) che catturano le prestazioni del sistema realizzato. Chiarisci quali sono gli indici di stima
e come sono calcolati. Inserisci un breve commento per ogni grafico.

\section{Conclusioni}
Conclusioni, possibili sviluppi futuri e limitazioni del progetto realizzato

%% Inserisci bibliografia dei lavori citati (consiglio l'utilizzo di bibtex)
\bibliographystyle{plain}
\begin{thebibliography}{15}

\bibitem{NomeRiferimento}
\newblock{Lista Autori} 
\newblock{Titolo Lavoro}
\newblock{\textit{Nome Rivista o Convegno}}, pagine, anno pubblicazione.

\end{thebibliography}
\end{document}